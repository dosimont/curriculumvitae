
\newcommand{\stw}{\phantom{ab}}

\section{Research Interest}
\subsection{Performance Analysis based on Trace Visualization}

\cventre{Scalable overviews for trace analysis}{\\
I designed \textbf{multidimensional overviews} that enable the global behavior representation of voluminous execution traces, up to a 100 GB size and composed of billions of events. These overviews are built upon data aggregation based on information theory measures, to reduce the trace complexity but keep the relevant information. The level of details can be adjusted to show phenomena of different scales.}\\
\cventre{Interaction and visual analytics}{\\
I designed several interaction mechanisms to perform an analysis flow from these overviews. Navigation is performed through data zooming, while keeping track of the position in the trace. \textbf{Statistics views} bring additional information on a selection. At last, the user can switch to more detailed views on subparts of the trace to refine the analysis.}\\
\cventre{Software design and implementation of these methods}{\\
I designed and implemented the visualization and interaction methods in the Eclipse plug-in Ocelotl. The aggregation is performed by an external library, LPAggreg.
\begin{itemize}
\item \textbf{Ocelotl} - \emph{lead developper} - \link{http://soctrace-inria.github.io/ocelotl}
\item \textbf{LPAggreg} library - \emph{lead developper}- \link{http://github.com/dosimont/lpaggreg}
\end{itemize}
}
\subsection{Trace and Tool Management}
\cventre{}{
I am receptive to the \textbf{trace storage and management} problematics, as well as the issues related to the communication between tools to perform an \textbf{analysis flow}. I am involved in the development of a trace analysis infrastructure that uses relational data bases and a generic data model.
\begin{itemize}
\item \textbf{Framesoc} - \emph{contributor} - \link{http://soctrace-inria.github.io/framesoc}
\end{itemize}
}
\subsection{Information Visualization}
\cventre{}{In a general way, the methods I propose can be applied in other fields. I am interested to \textbf{enlarge my researches} to \textbf{other applications} whose structure and behavior over time are complex, such as biology, economics, finance...}


\section{Skills}
% \cventr{Trace Visualization, Performance Analysis, Big Data}{
% \begin{itemize}
%  \item Scalable trace visualizations based on data aggregation 
% techniques built upon information theory measures
% \end{itemize}
% }
% \cventr{Communication}{
% \begin{itemize}
% \item 3 papers in international conferences, 3 papers in workshops, 3 research reports
% \item Talks in several countries (Japan, Germany, Italy, Spain, France)
% \end{itemize}
% }

% \cventr{Project Involvement}{
% \begin{itemize}
% \item Interaction with industrial partners, occasional 
% consulting 
% (STMicroelectronics)
% \item Advising a research engineer
% \end{itemize}
% }
% \cventro{Teaching}{\begin{itemize}\item Unix, Algorithmic, HCI, C/C++ 
% and Java 
% practicals (128 hours) \\at \emph{Polytech'Grenoble Engineering 
% School}\end{itemize}}
% 
% \cventr{Parallel Systems}{\begin{itemize}\item MPI, Computing Grid 
% Experiments, Tracing \end{itemize}
% }
\cventro{Programming}{\begin{itemize} \item 
C/C++, Java, Shell, Python, Shell Unix, ASM, SystemC, 
VHDL, HTML, \LaTeX, R \end{itemize}}
\cventro{Tools}{\begin{itemize} \item Eclipse, Git/Github, SVN, GCC, Make \end{itemize}}
%\section{Computer Science Skills}
% \cvline{Systems}{\small \textbf{Unix/Linux} (Fedora, Ubuntu), \textbf{Windows} 
% (7, Vista, XP)}
% \cvline{Architectures}{\small \textbf{ARM} (Cortex A8, Cortex A9), 
% \textbf{MIPS}, \textbf{x86}, \textbf{x64}}
% \cvline{Softwares}{\small \textbf{Eclipse}, Xilinx, Altera Quartus, Modelsim, 
% PSpice, Matlab, Office, Visio}

 %, ASM, VHDL, VHDL-AMS, HTML, Lustre, Esterel}
% \cvline{Key skills}{\small \textbf \textbf{Embedded systems architecture} 
% (hardware, software), \textbf{POSIX}, middlewares, \textbf{virtual prototyping}, 
% drivers, FPGA, VLSI, consumption}