
\newcommand{\stw}{\phantom{ab}}

\section{Research Interest}
\subsection{Performance Analysis based on Trace Visualization}

\cventre{Scalable overviews for trace analysis}{\\
I designed \textbf{multidimensional overviews} that enable the behavior representation of voluminous traces, provided by \textbf{embedded} or \textbf{parallel} application executions, up to 100 GB and composed of billions of events. They are built upon information \textbf{theory measures}, in order to reduce the representation complexity while keeping the relevant information about the phenomena contained in the trace.}\\
\cventre{Interaction and visual analytics}{\\
I designed several interaction mechanisms to perform an analysis flow from these overviews. Navigating in the trace is possible through \textbf{zooming features} that keep also track of the position in the trace. \textbf{Statistics views} bring additional information on the selection. At last, switching to \textbf{more detailed views}, such as the Gantt chart, enables to refine the analysis and focus on a particular part of the trace.}\\
\cventre{Software design and implementation of these methods}{
I designed and implemented the visualization and interaction methods in the Eclipse plug-in Ocelotl. The aggregation is performed by an external library, LPAggreg.
\begin{itemize}
\item \textbf{Ocelotl} - \emph{lead developper} - \link{http://soctrace-inria.github.io/ocelotl}
\item \textbf{LPAggreg} library - \emph{lead developper}- \link{http://github.com/dosimont/lpaggreg}
\end{itemize}
}
\subsection{Trace and Tool Management}
\small I have also a sensibility to the trace storage and management problematics, as well as the issues related to the communication between tools to perform an analysis flow. I participated to the development of Framesoc, a trace analysis infrastructure that manages traces thanks to relational data bases and a generic data model.
\begin{itemize}
\item \textbf{Framesoc} - \emph{contributor} - \link{http://soctrace-inria.github.io/framesoc}
\end{itemize}



\section{Skills}
% \cventr{Trace Visualization, Performance Analysis, Big Data}{
% \begin{itemize}
%  \item Scalable trace visualizations based on data aggregation 
% techniques built upon information theory measures
% \end{itemize}
% }
% \cventr{Communication}{
% \begin{itemize}
% \item 3 papers in international conferences, 3 papers in workshops, 3 research reports
% \item Talks in several countries (Japan, Germany, Italy, Spain, France)
% \end{itemize}
% }

% \cventr{Project Involvement}{
% \begin{itemize}
% \item Interaction with industrial partners, occasional 
% consulting 
% (STMicroelectronics)
% \item Advising a research engineer
% \end{itemize}
% }
\cventro{Teaching}{\begin{itemize}\item Unix, Algorithmic, HCI, C/C++ 
and Java 
practicals (128 hours) \\at \emph{Polytech'Grenoble Engineering 
School}\end{itemize}}
% 
% \cventr{Parallel Systems}{\begin{itemize}\item MPI, Computing Grid 
% Experiments, Tracing \end{itemize}
% }
\cventro{Programming}{\begin{itemize} \item 
C/C++, Java, Shell, Python, Shell Unix, ASM, SystemC, 
VHDL, HTML, \LaTeX, R \end{itemize}}
\cventro{Tools}{\begin{itemize} \item Eclipse, Git/Github, SVN, GCC, Make \end{itemize}}
%\section{Computer Science Skills}
% \cvline{Systems}{\small \textbf{Unix/Linux} (Fedora, Ubuntu), \textbf{Windows} 
% (7, Vista, XP)}
% \cvline{Architectures}{\small \textbf{ARM} (Cortex A8, Cortex A9), 
% \textbf{MIPS}, \textbf{x86}, \textbf{x64}}
% \cvline{Softwares}{\small \textbf{Eclipse}, Xilinx, Altera Quartus, Modelsim, 
% PSpice, Matlab, Office, Visio}

 %, ASM, VHDL, VHDL-AMS, HTML, Lustre, Esterel}
% \cvline{Key skills}{\small \textbf \textbf{Embedded systems architecture} 
% (hardware, software), \textbf{POSIX}, middlewares, \textbf{virtual prototyping}, 
% drivers, FPGA, VLSI, consumption}